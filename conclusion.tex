In this paper, we presented \emph{static pair-wise lock set analysis}, a novel sound verification technique for proving the absence of data races in device drivers. In the heart of our approach, we combine static lock set analysis with sound sequentialisation and automated theorem proving. Our lock set analysis enforces the locking discipline that two threads accessing the same shared resource must hold at least one common lock, and reports a potential data race if this discipline is violated. We have prototyped this approach in \textsc{Whoop}, a practical tool for automatic concurrency verification of Linux drivers.

Compared to traditional data race detection tools that typically attempt to explore as many thread interleavings as feasible, and thus face scalability issues in the presence of realistic concurrent programs, \textsc{Whoop} not only entirely avoids reasoning about thread interleavings, but also allows the reuse of existing successful sequential verification techniques. The main limitation of our approach is that it can potentially report many false positives, as we over-approximate the device driver shared state. To tackle this problem, we plan to investigate (i) invariant generation for taming our coarse abstraction and (ii) counterexample feasibility checking to evaluate if a reported bug is real or spurious.

Finally, we plan to perform a large-scale evaluation of the technique against the state-of-the-art as soon as the prototype is complete. For this purpose, we are already gathering a large number of publicly available Linux device drivers.