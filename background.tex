\noindent\textbf{Concurrency in Device Drivers }
%
Modern operating systems address the demand for responsiveness and performance in device drivers by providing multiple sources of concurrency~\cite{corbet2005linux}: an arbitrary number of entry points from the same driver can be invoked concurrently; a running driver process can block, causing the driver to switch execution to another process; and hardware interrupts can be handled concurrently.  These forms of concurrent execution are prone to \emph{data races}.

\begin{definition}
\label{definition:datarace}
A \emph{data race} occurs when two distinct threads access a shared memory location, at least one of the accesses is a write access, at least one of the accesses is non-atomic, and there is no mechanism in place to prevent these accesses from being simultaneous.
\end{definition}

Fig.~\ref{fig:data_race_example} shows a racy entry point, \texttt{nvram\_llseek}, in the generic\_nvram Linux 4.0 driver. The driver can invoke the entry point concurrently from two threads, with the same \texttt{file} struct as a parameter. This can lead to multiple possible data races because the threads can access the \texttt{f\_pos} field of \texttt{file} in arbitrary order. Our tool, \whoop (see \S\ref{whoop}), was able to find these races automatically (see \S\ref{evaluation}).

\begin{figure}[t]
\begin{lstlisting}
static loff_t nvram_llseek(struct file *file,
    loff_t offset, int origin) {
  switch (origin) {
    ...
    case 1: offset += file->f_pos; break; // racy
    ...
  }
  if (offset < 0) return -EINVAL;
  file->f_pos = offset; // racy
  return file->f_pos; // racy
}
\end{lstlisting}
\vspace{-2mm}
\caption{Racy entry point in the generic\_nvram Linux 4.0 driver.}
\label{fig:data_race_example}
%\vspace{-2mm}
\end{figure}

The most common method for avoiding races is by protecting sets of program statements that access a shared resource with \emph{locks}, forming \emph{critical sections}.  Fig.~\ref{fig:lock_example} shows how to use locking to eliminate the races in Fig.~\ref{fig:data_race_example}.
%
%Because the \texttt{return} statement can potentially race on the \texttt{f\_pos} field, we store the result in a temporary variable \texttt{res} inside the critical section.
%
Careless use of locks has many well-known pitfalls~\cite{sutter2005software}: coarse-grained locking can hurt performance as it limits the opportunity for concurrency, while fine-grained locking can easily lead to deadlocks. Although the Linux kernel provides sophisticated lock-free synchronization mechanisms~\cite[p.\ 123]{corbet2005linux}, such as atomic read-modify-write operations, in this work we focus on locks as they are widely used.\footnote{We treat lock-free operations soundly by regarding them as not providing any protection between threads; this can lead to false alarms.}

\begin{figure}[t]
\begin{lstlisting}
static loff_t nvram_llseek(struct file *file,
    loff_t offset, int origin) {
  mutex_lock(&nvram_mutex); // lock
  switch (origin) {
    ...
    case 1: offset += file->f_pos; break;
    ...
  }
  if (offset < 0) {
    mutex_unlock(&nvram_mutex); // unlock
    return -EINVAL;
  }
  file->f_pos = offset;
  loff_t res = file->f_pos; // store result
  mutex_unlock(&nvram_mutex); // unlock
  return res;
}
\end{lstlisting}
\vspace{-2mm}
\caption{Introducing a lock to eliminate the races in the previous example.}
\label{fig:lock_example}
%\vspace{-2mm}
\end{figure}

\noindent\textbf{Lockset Analysis }
%
Lockset analysis is a lightweight race detection method proposed in the context of Eraser~\cite{savage1997eraser}, a dynamic data race detector.  The idea is to track the set of locks (i.e.\ \emph{lockset}) that are \emph{consistently} used to protect a memory location during program execution. An empty lockset suggests that a memory location \emph{may} be accessed simultaneously by two or more threads. Consequently, the analysis reports a \emph{potential} race on that memory location.

Lockset analysis for a concurrent program starts by creating a \emph{current} lockset $\mathit{CLS}_T$ for each thread $T$ of the program, and a lockset $\mathit{LS}_s$ for each shared variable $s$ used in the program. Initially, every $\mathit{CLS}_T$ is empty because the threads do not hold any locks on program start. In addition, every $\mathit{LS}_s$ is initialized to the set of all locks manipulated by the program since initially each access to $s$ is (vacuously) protected by every lock. The program is executed as usual (with threads scheduled according to the OS scheduler), except that instrumentation is added to update locksets as follows.
%
After each \emph{lock} and \emph{unlock} operation by $T$, $\mathit{CLS}_T$ is updated to reflect the locks currently held by $T$.
%
When $T$ accesses $s$, $\mathit{LS}_s$ is updated to the intersection of $\mathit{LS}_s$ with $\mathit{CLS}_T$, which removes any locks that are not common to the two locksets.
%
If $\mathit{LS}_s$ becomes empty as a result, a warning is issued that the access to $s$ may be unprotected.

Fig.~\ref{fig:locksets} shows an example of applying lockset analysis to a concurrent program consisting of two threads $T_1$ and $T_2$, both accessing a global variable $A$. Initially, $\mathit{LS}_A$, which is the lockset for A, contains all possible locks in the program: $M$ and $N$. During execution of $T_1$, the thread writes $A$ without holding $N$, and thus $N$ is removed from $\mathit{LS}_A$. Next, during execution of $T_2$, the thread writes $A$ without holding $M$, and thus $\mathit{LS}_A$ becomes empty. As a result, a warning is reported because the two threads do not consistently protect $A$.

In contrast to more sophisticated race analyses that encode a \emph{happens-before} relation between threads~\cite{lamport1978time} (e.g.\ using vector clocks), lockset analysis is easy to implement, lightweight, and thus has the potential to scale well.  The technique, though, can report false alarms since a violation of the locking discipline does not always correspond to a real data race~\cite{savage1997eraser, pozniansky2003efficient, o2003hybrid, elmas2007goldilocks, flanagan2009fasttrack}. Furthermore, the code coverage in dynamic lockset analyzers, such as Eraser, is limited by the execution paths that are explored under a given scheduler.

To counter the above limitations, techniques such as Locksmith~\cite{pratikakis2006locksmith} and RELAY~\cite{voung2007relay} have explored the idea of applying lockset analysis statically, using dataflow analysis. In this paper, we present a novel symbolic lockset analysis method that involves generating verification conditions, which are then discharged to a theorem prover.

\begin{figure}[t]
\centering
\includegraphics[width=1\linewidth]{img/lockset.pdf}
%\vspace{-5mm}
\caption{Applying lockset analysis on a concurrent program.}
\label{fig:locksets}
%\vspace{-3mm}
\end{figure}
