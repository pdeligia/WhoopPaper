\whoop is a sound but potentially imprecise static data race analyser. In our experience, developers appreciate soundness but do not want to get overwhelmed by countless false positive reports. Towards this we allow an external precise bug-finder to be plugged in the \whoop infrastructure. The race-related information generated by \whoop can then be used to accelerate the bug-finder a bug-finder for concurrent programs that can be easily plugged in our infrastructure. Currently we only support \corral, a state-of-the-art bug-finder used in Microsoft to analyse Windows device drivers. We chose \corral because it is open-source and because it accepts Boogie IVL, similar to \whoop. Our technique though should be able to accelerate an arbitrary bug-finder for concurrent programs.

Sound partial order reduction

Yield instrumentation

Granularities:

Full-force

Coarse

MemoryRegion

Speeding up Corral

%\textsc{Whoop} is currently working on an unmodified network device driver (the RealTek r8169 which is part of the official Linux kernel distribution). Our immediate future work includes applying invariant generation to tame the effect of our coarse over-approximation. We also plan to introduce counterexample feasibility checking to evaluate if a reported bug is real or spurious. Towards this, we plan to feed a counterexample generated by \textsc{Whoop} into a precise bug-finder, such as Corral~\cite{lal2012corral}, which will be guided only to execution paths that may manifest the bug, to either confirm it or discard it as a false positive.