Statically analyzing concurrent programs to detect races is a promising alternative to dynamic analyzers, which commonly face code coverage issues as they rely on a (controlled) scheduler for exploring execution paths~\cite{musuvathi2008finding}. Warlock~\cite{sterling1993warlock} and LockLint~\cite{oracle2010locklint} are notable examples of static race analyzers. However, both tools have limited applicability, as they heavily rely on user annotations. \textsc{Whoop}, on the other hand, does not require any source code modifications, and thus can be applied with minimal effort.

Most related to our work are the static lockset analyzers RELAY~\cite{voung2007relay} and Locksmith~\cite{pratikakis2006locksmith}. Both tools, though, have significant limitations. The authors of RELAY found 5022 warnings when analyzing the Linux kernel, with only 25 of them being true data races. To limit these false positives, RELAY employs post-analysis unsound filters, but these can also filter out true races. Although Locksmith successfully detected data races in several small Pthreads applications and 7 medium-sized Linux device drivers, it also reported a significant number of false positives. The authors also reported that Locksmith was unable to run on several large programs, showcasing its limited scalability. In contrast, \whoop is based on novel over-approximation techniques and uses modern SMT solvers to accelerate state-of-the-art concurrency bug-finders, such as \corral, for precise \emph{and} scalable analysis.