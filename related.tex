Statically analyzing concurrent programs to detect data races is a promising alternative to dynamic techniques, which commonly face scalability and code coverage issues as they typically rely on a (controlled) scheduler to explore execution paths. Notable previous work includes the static analyzers Warlock~\cite{sterling1993warlock}, ESC~\cite{detlefs1998extended} and LockLint~\cite{oracle2010locklint}. However, all these three tools heavily rely on user annotations, and thus require a lot of manual effort that limits their applicability. \textsc{Whoop}, on the other hand, does not require any source code modifications, and thus can be applied with zero effort.

Most related to our work are the static lockset analyzers RELAY~\cite{voung2007relay} and Locksmith~\cite{pratikakis2006locksmith}. Both tools, though, have significant limitations. The authors of RELAY found 5022 warnings when analyzing the Linux kernel, with only 25 of them being true data races. To limit these false positives, RELAY employs post-analysis unsound filters, but these can also filter out true races. Although Locksmith successfully detected data races in several small Pthreads applications and 7 medium-sized Linux device drivers, it also reported a significant number of false positives. The authors also reported that Locksmith was unable to run on several large programs, showcasing its limited scalability.