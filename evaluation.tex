\newcommand{\colspacing}{\hspace{1.8em}}
\begin{table}[t]
\small
\centering
\setlength{\tabcolsep}{0.3em}
\caption{Program statistics and race-checking results from applying \whoop and \corral on our benchmarks.}
\label{tab:stats}
\begin{tabular}{l rrr rr r}
\centering
& & & & \multicolumn{2}{c}{\textbf{\whoop}}
& \textbf{\corral}\\
\cmidrule(lr){5-6}
\cmidrule(lr){7-7}

& & & & \multicolumn{1}{r}{\textbf{\#Racy}}
& \multicolumn{1}{r}{\textbf{\#Racy}}
& \multicolumn{1}{r}{\textbf{\#Races}}\\

\textbf{Benchmarks}
& \textbf{LoC}
& \textbf{\#Pairs}
& \textbf{\#MRs}
& \multicolumn{1}{r}{\textbf{Pairs}}
& \multicolumn{1}{r}{\textbf{MRs}}
& \multicolumn{1}{r}{\textbf{Found}}\\[0.3em]

\toprule

generic\_nvram
& 283
& 14
& 39
& \multicolumn{1}{r}{7}
& \multicolumn{1}{r}{2}
& 4\\

pc8736x\_gpio
& 354
& 27
& 55
& \multicolumn{1}{r}{13}
& \multicolumn{1}{r}{6}
& 5\\

machzwd
& 457
& 10
& 49
& \multicolumn{1}{r}{6}
& \multicolumn{1}{r}{3}
& 1\\

ssu100
& 568
& 7
& 27
& \multicolumn{1}{r}{\xmark}
& \multicolumn{1}{r}{\xmark}
& \xmark\\

intel\_scu\_wd
& 632
& 10
& 45
& \multicolumn{1}{r}{5}
& \multicolumn{1}{r}{1}
& 2\\

ds1286
& 635
& 15
& 49
& \multicolumn{1}{r}{3}
& \multicolumn{1}{r}{3}
& \xmark\\

dtlk
& 750
& 21
& 53
& \multicolumn{1}{r}{10}
& \multicolumn{1}{r}{6}
& \xmark\\

fs3270
& 883
& 15
& 54
& \multicolumn{1}{r}{9}
& \multicolumn{1}{r}{1}
& \xmark\\

gdrom
& 890
& 94
& 41
& \multicolumn{1}{r}{21}
& \multicolumn{1}{r}{1}
& \xmark\\

swim
& 996
& 28
& 80
& \multicolumn{1}{r}{13}
& \multicolumn{1}{r}{8}
& 6\\

intel\_nfcsim
& 1272
& 10
& 24
& \multicolumn{1}{r}{10}
& \multicolumn{1}{r}{2}
& \xmark\\

ps3vram
& 1499
& 4
& 32
& \multicolumn{1}{r}{1}
& \multicolumn{1}{r}{1}
& \xmark\\

sonypi
& 1729
& 30
& 62
& \multicolumn{1}{r}{19}
& \multicolumn{1}{r}{4}
& 2\\

sx8
& 1751
& 2
& 47
& \multicolumn{1}{r}{1}
& \multicolumn{1}{r}{1}
& 1\\

8139too
& 2694
& 46
& 37
& \multicolumn{1}{r}{40}
& \multicolumn{1}{r}{4}
& \xmark\\

r8169
& 7205
& 192
& 50
& \multicolumn{1}{r}{88}
& \multicolumn{1}{r}{1}
& \xmark\\

\bottomrule
\end{tabular}
%\vspace{-3mm}
\end{table}

\begin{table*}[t]
\small
\centering
\setlength{\tabcolsep}{0.4em}
\caption{Comparison with different yield instrumentation granularities and context-switch bounds.}
\label{tab:results}
\begin{tabular}{l r rrrr rrrr rrrr}
\centering
& \textbf{\whoop}
& \multicolumn{4}{c}{\textbf{\corral}}
& \multicolumn{8}{c}{\textbf{\whoop + \corral}}\\
\cmidrule(lr){2-2}
\cmidrule(lr){3-6}
\cmidrule(lr){7-14}

& \multirow{2}{*}{\textbf{Time}}
& \multicolumn{4}{c}{\textbf{\yieldall\xspace --- Time (sec)}}
& \multicolumn{4}{c}{\textbf{\yieldcoarse\xspace --- Time (sec)}}
& \multicolumn{4}{c}{\textbf{\yieldmr\xspace --- Time (sec)}}\\
\cmidrule(lr){3-6}
\cmidrule(lr){7-10}
\cmidrule(lr){11-14}

\textbf{Benchmarks}
& \textbf{\textbf{(sec)}}
& \multicolumn{1}{r}{\textbf{\#Y}}
& \multicolumn{1}{r}{\textbf{csb = 2}}
& \multicolumn{1}{r}{\textbf{csb = 5}}
& \multicolumn{1}{r}{\textbf{csb = 9}}
& \multicolumn{1}{r}{\textbf{\#Y}}
& \multicolumn{1}{r}{\textbf{csb = 2}}
& \multicolumn{1}{r}{\textbf{csb = 5}}
& \multicolumn{1}{r}{\textbf{csb = 9}}
& \multicolumn{1}{r}{\textbf{\#Y}}
& \multicolumn{1}{r}{\textbf{csb = 2}}
& \multicolumn{1}{r}{\textbf{csb = 5}}
& \multicolumn{1}{r}{\textbf{csb = 9}}\\[0.3em]

\toprule

generic\_nvram
& \multicolumn{1}{r}{2.6}
& \multicolumn{1}{r}{92}
& \multicolumn{1}{r}{32.0}
& \multicolumn{1}{r}{67.7}
& \multicolumn{1}{r}{3555.4}
& \multicolumn{1}{r}{47}
& \multicolumn{1}{r}{17.7}
& \multicolumn{1}{r}{24.2}
& \multicolumn{1}{r}{132.1}
& \multicolumn{1}{r}{29}
& \multicolumn{1}{r}{13.5}
& \multicolumn{1}{r}{16.9}
& \multicolumn{1}{r}{49.3}\\

pc8736x\_gpio
& \multicolumn{1}{r}{4.7}
& \multicolumn{1}{r}{691}
& \multicolumn{1}{r}{167.7}
& \multicolumn{1}{r}{333.2}
& \multicolumn{1}{r}{22656.5}
& \multicolumn{1}{r}{500}
& \multicolumn{1}{r}{92.2}
& \multicolumn{1}{r}{432.4}
& \multicolumn{1}{r}{22556.0}
& \multicolumn{1}{r}{167}
& \multicolumn{1}{r}{41.3}
& \multicolumn{1}{r}{79.4}
& \multicolumn{1}{r}{1343.2}\\

machzwd
& \multicolumn{1}{r}{3.0}
& \multicolumn{1}{r}{104}
& \multicolumn{1}{r}{37.9}
& \multicolumn{1}{r}{45.6}
& \multicolumn{1}{r}{78.4}
& \multicolumn{1}{r}{51}
& \multicolumn{1}{r}{26.1}
& \multicolumn{1}{r}{31.2}
& \multicolumn{1}{r}{55.2}
& \multicolumn{1}{r}{22}
& \multicolumn{1}{r}{15.7}
& \multicolumn{1}{r}{17.3}
& \multicolumn{1}{r}{23.9}\\

ssu100
& \multicolumn{1}{r}{3.1}
& \multicolumn{1}{r}{82}
& \multicolumn{1}{r}{11.1}
& \multicolumn{1}{r}{13.8}
& \multicolumn{1}{r}{37.3}
& \multicolumn{1}{r}{\xmark}
& \multicolumn{1}{r}{3.1}
& \multicolumn{1}{r}{3.1}
& \multicolumn{1}{r}{3.2}
& \multicolumn{1}{r}{\xmark}
& \multicolumn{1}{r}{3.1}
& \multicolumn{1}{r}{3.1}
& \multicolumn{1}{r}{3.1}\\

intel\_scu\_wd
& \multicolumn{1}{r}{2.8}
& \multicolumn{1}{r}{314}
& \multicolumn{1}{r}{22.8}
& \multicolumn{1}{r}{130.3}
& \multicolumn{1}{r}{1571.7}
& \multicolumn{1}{r}{217}
& \multicolumn{1}{r}{14.7}
& \multicolumn{1}{r}{70.0}
& \multicolumn{1}{r}{748.0}
& \multicolumn{1}{r}{196}
& \multicolumn{1}{r}{13.5}
& \multicolumn{1}{r}{66.1}
& \multicolumn{1}{r}{689.3}\\

ds1286
& \multicolumn{1}{r}{}
& \multicolumn{1}{r}{513}
& \multicolumn{1}{r}{35.0}
& \multicolumn{1}{r}{40.2}
& \multicolumn{1}{r}{51.8}
& \multicolumn{1}{r}{245}
& \multicolumn{1}{r}{22.3}
& \multicolumn{1}{r}{25.7}
& \multicolumn{1}{r}{33.0}
& \multicolumn{1}{r}{129}
& \multicolumn{1}{r}{14.5}
& \multicolumn{1}{r}{16.1}
& \multicolumn{1}{r}{19.3}\\

dtlk
& \multicolumn{1}{r}{}
& \multicolumn{1}{r}{801}
& \multicolumn{1}{r}{182.6}
& \multicolumn{1}{r}{263.7}
& \multicolumn{1}{r}{793.0}
& \multicolumn{1}{r}{664}
& \multicolumn{1}{r}{91.2}
& \multicolumn{1}{r}{150.7}
& \multicolumn{1}{r}{633.2}
& \multicolumn{1}{r}{286}
& \multicolumn{1}{r}{41.6}
& \multicolumn{1}{r}{52.9}
& \multicolumn{1}{r}{104.6}\\

fs3270
& \multicolumn{1}{r}{}
& \multicolumn{1}{r}{321}
& \multicolumn{1}{r}{77.5}
& \multicolumn{1}{r}{438.8}
& \multicolumn{1}{r}{}
& \multicolumn{1}{r}{239}
& \multicolumn{1}{r}{62.6}
& \multicolumn{1}{r}{405.7}
& \multicolumn{1}{r}{}
& \multicolumn{1}{r}{211}
& \multicolumn{1}{r}{40.7}
& \multicolumn{1}{r}{295.3}
& \multicolumn{1}{r}{}\\

gdrom
& \multicolumn{1}{r}{}
& \multicolumn{1}{r}{2058}
& \multicolumn{1}{r}{388.0}
& \multicolumn{1}{r}{390.8}
& \multicolumn{1}{r}{392.1}
& \multicolumn{1}{r}{1143}
& \multicolumn{1}{r}{104.0}
& \multicolumn{1}{r}{105.5}
& \multicolumn{1}{r}{107.1}
& \multicolumn{1}{r}{812}
& \multicolumn{1}{r}{99.2}
& \multicolumn{1}{r}{102.4}
& \multicolumn{1}{r}{107.6}\\

swim
& \multicolumn{1}{r}{}
& \multicolumn{1}{r}{1270}
& \multicolumn{1}{r}{271.0}
& \multicolumn{1}{r}{}
& \multicolumn{1}{r}{}
& \multicolumn{1}{r}{996}
& \multicolumn{1}{r}{164.9}
& \multicolumn{1}{r}{}
& \multicolumn{1}{r}{}
& \multicolumn{1}{r}{805}
& \multicolumn{1}{r}{95.5}
& \multicolumn{1}{r}{}
& \multicolumn{1}{r}{}\\

intel\_nfcsim
& \multicolumn{1}{r}{}
& \multicolumn{1}{r}{732}
& \multicolumn{1}{r}{39.8}
& \multicolumn{1}{r}{85.0}
& \multicolumn{1}{r}{1539.9}
& \multicolumn{1}{r}{732}
& \multicolumn{1}{r}{44.4}
& \multicolumn{1}{r}{89.6}
& \multicolumn{1}{r}{1,543.4}
& \multicolumn{1}{r}{601}
& \multicolumn{1}{r}{21.0}
& \multicolumn{1}{r}{32.0}
& \multicolumn{1}{r}{278.0}\\

ps3vram
& \multicolumn{1}{r}{}
& \multicolumn{1}{r}{189}
& \multicolumn{1}{r}{99.6}
& \multicolumn{1}{r}{1816.7}
& \multicolumn{1}{r}{}
& \multicolumn{1}{r}{176}
& \multicolumn{1}{r}{96.2}
& \multicolumn{1}{r}{1807.3}
& \multicolumn{1}{r}{}
& \multicolumn{1}{r}{156}
& \multicolumn{1}{r}{55.8}
& \multicolumn{1}{r}{1834.6}
& \multicolumn{1}{r}{}\\

sonypi
& \multicolumn{1}{r}{}
& \multicolumn{1}{r}{1745}
& \multicolumn{1}{r}{1966.5}
& \multicolumn{1}{r}{}
& \multicolumn{1}{r}{}
& \multicolumn{1}{r}{1566}
& \multicolumn{1}{r}{1924.4}
& \multicolumn{1}{r}{}
& \multicolumn{1}{r}{}
& \multicolumn{1}{r}{1024}
& \multicolumn{1}{r}{906.0}
& \multicolumn{1}{r}{}
& \multicolumn{1}{r}{}\\

sx8
& \multicolumn{1}{r}{}
& \multicolumn{1}{r}{227}
& \multicolumn{1}{r}{14.4}
& \multicolumn{1}{r}{23.5}
& \multicolumn{1}{r}{699.8}
& \multicolumn{1}{r}{227}
& \multicolumn{1}{r}{16.1}
& \multicolumn{1}{r}{18.5}
& \multicolumn{1}{r}{24.6}
& \multicolumn{1}{r}{217}
& \multicolumn{1}{r}{15.9}
& \multicolumn{1}{r}{18.5}
& \multicolumn{1}{r}{24.3}\\

8139too
& \multicolumn{1}{r}{}
& \multicolumn{1}{r}{7151}
& \multicolumn{1}{r}{548.2}
& \multicolumn{1}{r}{2423.4}
& \multicolumn{1}{r}{12726.3}
& \multicolumn{1}{r}{6266}
& \multicolumn{1}{r}{474.7}
& \multicolumn{1}{r}{}
& \multicolumn{1}{r}{}
& \multicolumn{1}{r}{4964}
& \multicolumn{1}{r}{359.8}
& \multicolumn{1}{r}{}
& \multicolumn{1}{r}{}\\

r8169
& \multicolumn{1}{r}{228.1}
& \multicolumn{1}{r}{}
& \multicolumn{1}{r}{T.O.}
& \multicolumn{1}{r}{T.O.}
& \multicolumn{1}{r}{T.O.}
& \multicolumn{1}{r}{}
& \multicolumn{1}{r}{28755.2}
& \multicolumn{1}{r}{T.O.}
& \multicolumn{1}{r}{T.O.}
& \multicolumn{1}{r}{}
& \multicolumn{1}{r}{26939.4}
& \multicolumn{1}{r}{T.O.}
& \multicolumn{1}{r}{T.O.}\\

\bottomrule
\end{tabular}
%\vspace{-3mm}
\end{table*}

\begin{figure}
\centering
\includegraphics[width=.99\linewidth]{experiments/figures/yieldmr_vs_yieldall.pdf}
%\vspace{-2mm}
\caption{Scatter plot showing the runtime speedups that \corral achieves using \whoop with the \yieldmr instrumentation. The symbols $+$, $\circ$ and $\times$, represent a context-switch bound of 2, 5 and 9, respectively.}
\label{fig:plot}
%\vspace{-2mm}
\end{figure}

We performed experiments to validate the usefulness of the \whoop approach (\S\ref{whoop}) and its combination with \corral (\S\ref{corral}). We first present race-checking results from running \whoop and \corral on \sizeOfBenchmarks drivers taken from the Linux 4.0 kernel.\footnote{\url{https://www.kernel.org}} We then evaluate the runtime performance and scalability of \corral using different yield instrumentations and context-switch bounds. Our results demonstrate that \whoop can efficiently accelerate race-checking with \corral.

For reproducibility, we make all of our tools (including source) and benchmarks available online:

\begin{scriptsize}
\textbf{\texttt{\url{http://multicore.doc.ic.ac.uk/tools/Whoop/ASE2015}}}
\end{scriptsize}

\noindent\textbf{Experimental Setup }
%
We performed all experiments on a 3.40GHz Intel Core i7-2600 CPU with 16GB RAM running Ubuntu Linux 12.04.5 LTS, LLVM 3.5, SMACK 1.5, Z3~4.3.2, Boogie rev. 4192 and \corral rev. 534. We also used Mono~4.1.0 to run Boogie and \corral.

\noindent\textbf{Benchmarks }
%
We evaluate our methodology against \sizeOfBenchmarks drivers from the Linux 4.0 kernel. We chose non-trivial drivers of several types: block; char; ethernet; near field communication (nfc); universal serial bus (usb); and watchdog. For all these drivers, we had to understand their environment and manually model it; this required approximately two months of work.

\noindent\textbf{Race-Checking }
%
Table~\ref{tab:stats} presents statistics for our benchmarks: lines of code (LoC); number of possible entry point pairs (\#Pairs); number of SMACK memory regions (\#MRs); number of racy pairs (\#Racy Pairs) and number of racy memory regions (\#Racy MRs) reported by \whoop; and number of data races discovered by \corral using a context-switch bound (csb) of 2 (\#Races Found).\footnote{The number of racy memory regions can be less than the number of races found by \corral, because \whoop might find that a memory region is racy, but the same memory region might race in many points in the program.} Using a higher csb did not uncover any further races; this might mean that all races in our benchmarks can manifest with a csb of $\leq$ 2, or that \corral hit its recursion depth bound before discovering a deeper bug.

We can see that \whoop reports more races than \corral does. This is because \whoop employs an over-approximating shared state abstraction to conservatively model the effects of additional threads when analyzing an entry point pair, and because lockset analysis is inherently imprecise; both factors can lead to false positives.  On the other hand, \corral is precise, but can miss races because only a limited number of context-switches are considered.  Another issue with \corral is loop coverage due to unsound loop unrolling. To tackle this, we enable the built-in loop over-approximation described in previous work~\cite{lal2014powering}. This can potentially lead \corral to report false bugs, but we have not seen this in practice. Furthermore, when we apply \corral to a pair of entry points, we just check the specific pair and do not account for the effects of other threads (see \S\ref{corral}); this can also cause \corral to miss some races. Standalone \corral did not discover any data races that \whoop did not already report.  This is expected, as \whoop aims for soundness, and increases our confidence in the implementation of \whoop.

Most of the races that \whoop and \corral discovered fall into two categories. The first is about accessing a global counter (or flag) from concurrent entry points, without holding a lock. This might be for performance, and indeed a lot of the races we found might be benign. Even benign races, though, lead to undefined behavior according to the C standard, and it is well known that undefined behaviors can lead to unexpected results when combined with aggressive compiler optimizations. The second is about an entry point modifying a field of a struct (either global or passed as a parameter) without holding a lock. This can lead to a race if another entry point simultaneously accesses the same field of the same struct.

As an example of the second category, we found the following race in the generic\_nvram driver (see Fig.~\ref{fig:data_race_example}): the \texttt{llseek} entry point accesses the file offset \texttt{file->f\_pos} without a lock (\texttt{file} is passed as a parameter to \texttt{llseek}). This causes a race if the driver invokes \texttt{llseek} from another thread passing the same \texttt{file} object as a parameter.  We observed that another char driver, using the same APIs, \emph{does} use a mutex to protect the offset access in its \texttt{llseek} entry point, leading us to suspect that the race we found in generic\_nvram is not benign.  We have filed a bug report in the Linux kernel bug tracker and are awaiting a response.

\noindent\textbf{Accelerating \corral }
%
Table~\ref{tab:results} presents runtime results from using \whoop, \corral and \whoop + \corral to analyze our benchmarks, while Fig.~\ref{fig:plot} plots the runtime speedups that \corral achieves using \whoop with the \yieldmr instrumentation. All reported times are in seconds and averaged over three runs. \corral was configured with a time budget of 10 hours (T.O. denotes timeout) and a csb of 2, 5 and 9. Standalone \corral uses \yieldall, which instruments context-switches (i.e. \texttt{yield} statements) after all visible operations, while \whoop + \corral uses \yieldcoarse and \yieldmr, which instrument context-switches in a more fine-grained fashion (see \S\ref{corral}). The table also shows the number of context-switches per instrumentation (\#Y).

\whoop uses over-approximation to scale and, as expected, executes significantly faster than \corral in all our benchmarks. For example, \corral times out in all configurations (with and without \whoop) when trying to analyze the r8169 ethernet driver, while \whoop manages to analyze the driver in 144.5 seconds. We believe that the reason behind this is that r8169 has deeply-nested recursion in some of its entry points, which might cause \corral to get stuck. This is not an issue for \whoop, which uses procedure summarization. This shows that \whoop has value as a stand-alone analyzer.

Using the race-freedom guarantees from \whoop, we managed to significantly accelerate \corral in most of our benchmarks; the best results were achieved using \yieldmr. Fig.~\ref{fig:plot} shows that most speedups using \yieldmr are between 1.5$\times$ and 10$\times$; while in ssu100, and pc8736x\_gpio respectively, we achieved a speedup of 12$\times$, and 20$\times$ respectively, with a csb of 9. We noticed that a higher csb typically results into greater speedups when exploiting \whoop. This is expected, as a higher csb translates into a larger sequentialized program, and thus \whoop has more opportunities to reduce this sequentialization.
%
However, there are some cases where \whoop might slow down \corral. We noticed this in the sx8 driver: \whoop verified 46 out of 47 memory regions, but did not fully verify any of the pairs; \corral, on the other hand, analyzed the only two pairs of the driver in just 21.4 seconds (csb of 9). We believe that in this case the overhead of running \whoop outweighed the benefits of using \yieldmr.

\noindent\textbf{Other Tools }
%
We tried to compare \whoop with other similar approaches (see \S\ref{related}). However, we found this to be hard in practice: we downloaded Locksmith~\cite{pratikakis2006locksmith}, but could not get it to work with the 4.0 Linux drivers (the tool was last updated in 2007); we also could not find source code or binaries for \cite{kahlon2007fast, kahlon2009semantic, das2015section}.
