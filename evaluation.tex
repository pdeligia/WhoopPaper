\newcommand{\colspacing}{\hspace{1.8em}}
\begin{table}[t]
\small
\centering
\setlength{\tabcolsep}{0.3em}
\caption{Program statistics and data race detection results from running \whoop and \corral on our benchmarks. Corral uses a context switch bound of 2.}
\label{tab:stats}
\begin{tabular}{l rrr rr r}
\centering
& & & & \multicolumn{2}{c}{\textbf{\whoop}}
& \textbf{\corral}\\
\cmidrule(lr){5-6}
\cmidrule(lr){7-7}

& & & & \multicolumn{1}{r}{\textbf{\#Racy}}
& \multicolumn{1}{r}{\textbf{\#Racy}}
& \multicolumn{1}{r}{\textbf{\#Races}}\\

\textbf{Benchmarks}
& \textbf{LoC}
& \textbf{\#Pairs}
& \textbf{\#MRs}
& \multicolumn{1}{r}{\textbf{Pairs}}
& \multicolumn{1}{r}{\textbf{MRs}}
& \multicolumn{1}{r}{\textbf{Found}}\\[0.3em]

\toprule

generic\_nvram
& 283
& 14
& 39
& \multicolumn{1}{r}{7}
& \multicolumn{1}{r}{2}
& 4\\

pc8736x\_gpio
& 354
& 27
& 55
& \multicolumn{1}{r}{13}
& \multicolumn{1}{r}{6}
& 5\\

machzwd
& 457
& 10
& 49
& \multicolumn{1}{r}{6}
& \multicolumn{1}{r}{3}
& 1\\

ssu100
& 568
& 7
& 27
& \multicolumn{1}{r}{\xmark}
& \multicolumn{1}{r}{\xmark}
& \xmark\\

intel\_scu\_wd
& 632
& 10
& 45
& \multicolumn{1}{r}{5}
& \multicolumn{1}{r}{1}
& 2\\

ds1286
& 635
& 15
& 49
& \multicolumn{1}{r}{5}
& \multicolumn{1}{r}{3}
& \xmark\\

dtlk
& 750
& 21
& 53
& \multicolumn{1}{r}{10}
& \multicolumn{1}{r}{6}
& \xmark\\

fs3270
& 883
& 15
& 54
& \multicolumn{1}{r}{9}
& \multicolumn{1}{r}{1}
& \xmark\\

gdrom
& 890
& 94
& 41
& \multicolumn{1}{r}{21}
& \multicolumn{1}{r}{2}
& \xmark\\

swim
& 996
& 28
& 80
& \multicolumn{1}{r}{15}
& \multicolumn{1}{r}{7}
& 8\\

intel\_nfcsim
& 1272
& 10
& 24
& \multicolumn{1}{r}{10}
& \multicolumn{1}{r}{2}
& \xmark\\

ps3vram
& 1499
& 4
& 32
& \multicolumn{1}{r}{1}
& \multicolumn{1}{r}{1}
& \xmark\\

sonypi
& 1729
& 30
& 62
& \multicolumn{1}{r}{19}
& \multicolumn{1}{r}{4}
& 2\\

sx8
& 1751
& 2
& 47
& \multicolumn{1}{r}{2}
& \multicolumn{1}{r}{1}
& 1\\

8139too
& 2694
& 46
& 37
& \multicolumn{1}{r}{40}
& \multicolumn{1}{r}{4}
& \xmark\\

r8169
& 7205
& 192
& 50
& \multicolumn{1}{r}{88}
& \multicolumn{1}{r}{1}
& \xmark\\

\bottomrule
\end{tabular}
\end{table}

\begin{table*}[t]
\small
\centering
\caption{Runtime comparison with different yield instrumentation granularities and context-switch bounds (csb).}
\label{tab:results}
\begin{tabular}{l rrr rrr rr r r}
\centering
& \textbf{\whoop}
& \multicolumn{3}{c}{\textbf{\corral}}
& \multicolumn{6}{c}{\textbf{\whoop + \corral}}\\
\cmidrule(lr){2-2}
\cmidrule(lr){3-5}
\cmidrule(lr){6-11}

& \multirow{2}{*}{\textbf{Time}}
& \multicolumn{3}{c}{\textbf{\yieldall\xspace --- Time (sec)}}
& \multicolumn{3}{c}{\textbf{\yieldcoarse\xspace --- Time (sec)}}
& \multicolumn{3}{c}{\textbf{\yieldmr\xspace --- Time (sec)}}\\
\cmidrule(lr){3-5}
\cmidrule(lr){6-8}
\cmidrule(lr){9-11}

\textbf{Benchmarks}
& \textbf{\textbf{(sec)}}
& \multicolumn{1}{r}{\textbf{csb = 2}}
& \multicolumn{1}{r}{\textbf{csb = 5}}
& \multicolumn{1}{r}{\textbf{csb = 9}}
& \multicolumn{1}{r}{\textbf{csb = 2}}
& \multicolumn{1}{r}{\textbf{csb = 5}}
& \multicolumn{1}{r}{\textbf{csb = 9}}
& \multicolumn{1}{r}{\textbf{csb = 2}}
& \multicolumn{1}{r}{\textbf{csb = 5}}
& \multicolumn{1}{r}{\textbf{csb = 9}}\\[0.3em]

\toprule

generic\_nvram
& \multicolumn{1}{r}{2.7}
& \multicolumn{1}{r}{27.9}
& \multicolumn{1}{r}{38.4}
& \multicolumn{1}{r}{146.2}
& \multicolumn{1}{r}{17.6}
& \multicolumn{1}{r}{22.2}
& \multicolumn{1}{r}{90.8}
& \multicolumn{1}{r}{14.2}
& \multicolumn{1}{r}{16.5}
& \multicolumn{1}{r}{42.0}\\

pc8736x\_gpio
& \multicolumn{1}{r}{5.3}
& \multicolumn{1}{r}{145.6}
& \multicolumn{1}{r}{302.0}
& \multicolumn{1}{r}{}
& \multicolumn{1}{r}{89.5}
& \multicolumn{1}{r}{229.5}
& \multicolumn{1}{r}{}
& \multicolumn{1}{r}{41.7}
& \multicolumn{1}{r}{56.9}
& \multicolumn{1}{r}{426.6}\\

machzwd
& \multicolumn{1}{r}{}
& \multicolumn{1}{r}{}
& \multicolumn{1}{r}{}
& \multicolumn{1}{r}{}
& \multicolumn{1}{r}{}
& \multicolumn{1}{r}{}
& \multicolumn{1}{r}{}
& \multicolumn{1}{r}{}
& \multicolumn{1}{r}{}
& \multicolumn{1}{r}{}\\

ssu100
& \multicolumn{1}{r}{}
& \multicolumn{1}{r}{}
& \multicolumn{1}{r}{}
& \multicolumn{1}{r}{}
& \multicolumn{1}{r}{}
& \multicolumn{1}{r}{}
& \multicolumn{1}{r}{}
& \multicolumn{1}{r}{}
& \multicolumn{1}{r}{}
& \multicolumn{1}{r}{}\\

intel\_scu\_wd
& \multicolumn{1}{r}{}
& \multicolumn{1}{r}{}
& \multicolumn{1}{r}{}
& \multicolumn{1}{r}{}
& \multicolumn{1}{r}{}
& \multicolumn{1}{r}{}
& \multicolumn{1}{r}{}
& \multicolumn{1}{r}{}
& \multicolumn{1}{r}{}
& \multicolumn{1}{r}{}\\

ds1286
& \multicolumn{1}{r}{}
& \multicolumn{1}{r}{}
& \multicolumn{1}{r}{}
& \multicolumn{1}{r}{}
& \multicolumn{1}{r}{}
& \multicolumn{1}{r}{}
& \multicolumn{1}{r}{}
& \multicolumn{1}{r}{}
& \multicolumn{1}{r}{}
& \multicolumn{1}{r}{}\\

dtlk
& \multicolumn{1}{r}{}
& \multicolumn{1}{r}{}
& \multicolumn{1}{r}{}
& \multicolumn{1}{r}{}
& \multicolumn{1}{r}{}
& \multicolumn{1}{r}{}
& \multicolumn{1}{r}{}
& \multicolumn{1}{r}{}
& \multicolumn{1}{r}{}
& \multicolumn{1}{r}{}\\

fs3270
& \multicolumn{1}{r}{}
& \multicolumn{1}{r}{}
& \multicolumn{1}{r}{}
& \multicolumn{1}{r}{}
& \multicolumn{1}{r}{}
& \multicolumn{1}{r}{}
& \multicolumn{1}{r}{}
& \multicolumn{1}{r}{}
& \multicolumn{1}{r}{}
& \multicolumn{1}{r}{}\\

gdrom
& \multicolumn{1}{r}{}
& \multicolumn{1}{r}{}
& \multicolumn{1}{r}{}
& \multicolumn{1}{r}{}
& \multicolumn{1}{r}{}
& \multicolumn{1}{r}{}
& \multicolumn{1}{r}{}
& \multicolumn{1}{r}{}
& \multicolumn{1}{r}{}
& \multicolumn{1}{r}{}\\

swim
& \multicolumn{1}{r}{}
& \multicolumn{1}{r}{}
& \multicolumn{1}{r}{}
& \multicolumn{1}{r}{}
& \multicolumn{1}{r}{}
& \multicolumn{1}{r}{}
& \multicolumn{1}{r}{}
& \multicolumn{1}{r}{}
& \multicolumn{1}{r}{}
& \multicolumn{1}{r}{}\\

intel\_nfcsim
& \multicolumn{1}{r}{}
& \multicolumn{1}{r}{}
& \multicolumn{1}{r}{}
& \multicolumn{1}{r}{}
& \multicolumn{1}{r}{}
& \multicolumn{1}{r}{}
& \multicolumn{1}{r}{}
& \multicolumn{1}{r}{}
& \multicolumn{1}{r}{}
& \multicolumn{1}{r}{}\\

ps3vram
& \multicolumn{1}{r}{}
& \multicolumn{1}{r}{}
& \multicolumn{1}{r}{}
& \multicolumn{1}{r}{}
& \multicolumn{1}{r}{}
& \multicolumn{1}{r}{}
& \multicolumn{1}{r}{}
& \multicolumn{1}{r}{}
& \multicolumn{1}{r}{}
& \multicolumn{1}{r}{}\\

sonypi
& \multicolumn{1}{r}{}
& \multicolumn{1}{r}{}
& \multicolumn{1}{r}{}
& \multicolumn{1}{r}{}
& \multicolumn{1}{r}{}
& \multicolumn{1}{r}{}
& \multicolumn{1}{r}{}
& \multicolumn{1}{r}{}
& \multicolumn{1}{r}{}
& \multicolumn{1}{r}{}\\

sx8
& \multicolumn{1}{r}{}
& \multicolumn{1}{r}{}
& \multicolumn{1}{r}{}
& \multicolumn{1}{r}{}
& \multicolumn{1}{r}{}
& \multicolumn{1}{r}{}
& \multicolumn{1}{r}{}
& \multicolumn{1}{r}{}
& \multicolumn{1}{r}{}
& \multicolumn{1}{r}{}\\

8139too
& \multicolumn{1}{r}{}
& \multicolumn{1}{r}{}
& \multicolumn{1}{r}{}
& \multicolumn{1}{r}{}
& \multicolumn{1}{r}{}
& \multicolumn{1}{r}{}
& \multicolumn{1}{r}{}
& \multicolumn{1}{r}{}
& \multicolumn{1}{r}{}
& \multicolumn{1}{r}{}\\

r8169
& \multicolumn{1}{r}{228.1}
& \multicolumn{1}{r}{T.O.}
& \multicolumn{1}{r}{T.O.}
& \multicolumn{1}{r}{T.O.}
& \multicolumn{1}{r}{28755.2}
& \multicolumn{1}{r}{T.O.}
& \multicolumn{1}{r}{T.O.}
& \multicolumn{1}{r}{26939.4}
& \multicolumn{1}{r}{T.O.}
& \multicolumn{1}{r}{T.O.}\\

\bottomrule
\end{tabular}
\end{table*}

%\begin{table*}[t]
%\small
%\centering
%\caption{Program statistics, analysis and scalability results. Corral uses a context switch bound of 2.}
%\label{tab:races}
%\begin{tabular}{l rrr rrr rr r r}
%\centering
%& & & & \multicolumn{2}{c}{\textbf{\whoop}}
& \multicolumn{2}{c}{\textbf{\corral}}
& \multicolumn{2}{c}{\textbf{\whoop + \corral}}\\
\cmidrule(lr){5-6}
\cmidrule(lr){7-8}
\cmidrule(lr){9-10}

& & & & \multirow{2}{*}{\textbf{\#Racy}}
& & \multirow{2}{*}{\textbf{\#Races}}
& \multicolumn{1}{c}{\textbf{\yieldall}}
& \multicolumn{1}{c}{\textbf{\yieldcoarse}}
& \multicolumn{1}{c}{\textbf{\yieldmr}}\\
\cmidrule(lr){8-8}
\cmidrule(lr){9-9}
\cmidrule(lr){10-10}

\textbf{Benchmarks}
& \textbf{LoC}
& \textbf{\#MR}
& \textbf{\#Pairs}
& \textbf{Pairs}
& \textbf{\textbf{Time (s)}}
& \multicolumn{1}{r}{\textbf{Found?}}
& \multicolumn{1}{r}{\textbf{Time (s)}}
& \multicolumn{1}{r}{\textbf{Time (s)}}
& \multicolumn{1}{r}{\textbf{Time (s)}}\\[0.3em]

\toprule

generic\_nvram
& 152
& 25
& 14
& \multicolumn{1}{r}{6}
& \multicolumn{1}{r}{3.15}
& \multicolumn{1}{r}{\xmark}
& \multicolumn{1}{r}{191.36}
& \multicolumn{1}{r}{85.61}
& \multicolumn{1}{r}{34.39}\\

efirtc
& 420
& 35
& 6
& \multicolumn{1}{r}{2}
& \multicolumn{1}{r}{3.25}
& \multicolumn{1}{r}{\xmark}
& \multicolumn{1}{r}{202.31}
& \multicolumn{1}{r}{91.69}
& \multicolumn{1}{r}{20.36}\\

machzwd
& 457
& 49
& 10
& \multicolumn{1}{r}{9}
& \multicolumn{1}{r}{6.07}
& \multicolumn{1}{r}{1}
& \multicolumn{1}{r}{121.184}
& \multicolumn{1}{r}{125.55}
& \multicolumn{1}{r}{44.59}\\

intel\_scu\_wtchdg
& 568
& 25
& 10
& \multicolumn{1}{r}{5}
& \multicolumn{1}{r}{3.19}
& \multicolumn{1}{r}{1}
& \multicolumn{1}{r}{139.22}
& \multicolumn{1}{r}{78.73}
& \multicolumn{1}{r}{36.73}\\

ds1286
& 581
& 40
& 15
& \multicolumn{1}{r}{3}
& \multicolumn{1}{r}{4.51}
& \multicolumn{1}{r}{4}
& \multicolumn{1}{r}{553.48}
& \multicolumn{1}{r}{114.68}
& \multicolumn{1}{r}{31.60}\\

dtlk
& 656
& 37
& 21
& \multicolumn{1}{r}{10}
& \multicolumn{1}{r}{6.51}
& \multicolumn{1}{r}{\xmark}
& \multicolumn{1}{r}{239.517}
& \multicolumn{1}{r}{115.47}
& \multicolumn{1}{r}{79.05}\\

minix
& 690
& 34
& 90
& \multicolumn{1}{r}{50}
& \multicolumn{1}{r}{29.92}
& \multicolumn{1}{r}{20}
& \multicolumn{1}{r}{}
& \multicolumn{1}{r}{}
& \multicolumn{1}{r}{357.66}\\

aztcd
& 2496
& 85
& 14
& \multicolumn{1}{r}{13}
& \multicolumn{1}{r}{}
& \multicolumn{1}{r}{}
& \multicolumn{1}{r}{}
& \multicolumn{1}{r}{}
& \multicolumn{1}{r}{}\\

8139too
& 2694
& 
& 
& \multicolumn{1}{r}{}
& \multicolumn{1}{r}{}
& \multicolumn{1}{r}{}
& \multicolumn{1}{r}{}
& \multicolumn{1}{r}{}
& \multicolumn{1}{r}{}\\

r8169
& 7205
& 
& 
& \multicolumn{1}{r}{}
& \multicolumn{1}{r}{}
& \multicolumn{1}{r}{}
& \multicolumn{1}{r}{}
& \multicolumn{1}{r}{}
& \multicolumn{1}{r}{}\\[0.2em]
%\end{tabular}
%\end{table*}
%
%\begin{table*}[t]
%\small
%\centering
%\caption{Analysis and scalability results. Corral uses a context switch bound of 5 and 9, respectively.}
%\label{tab:races2}
%\begin{tabular}{l rr r r rr r r}
%\centering
%\input{experiments/tables/granularity2.tex}
%\end{tabular}
%\end{table*}

\subsection{Experimental Setup}
\label{eval:setup}

All experiments were performed on a ... machine running ..., LLVM 3.5; SMACK 1.5; Z3 4.3.2; Boogie rev. 4192 and \corral rev. 534.

\subsection{Benchmarks}
\label{eval:benchmarks}

We evaluate our methodology against \sizeOfBenchmarks drivers taken from the 4.0 Linux kernel. \PDComment{Say something more about the benchmarks}

Table~\ref{tab:stats} presents statistics for all our benchmarks: lines of code (LoC); number of entry point pairs (\#Pairs); number of SMACK memory regions (\#MRs); number of racy pairs that \whoop identified (\#Racy Pairs); number of racy memory regions that \whoop reported (\#Racy MRs); and number of races that \corral discovered (\#Races Found).

\subsection{Context-Switch Granularity of Yield Instrumentation}
\label{eval:granularity}

We evaluate the scalability of our acceleration technique with respect to our benchmark sets (Table~\ref{tab:results}). All reported times are in seconds and averages of three runs. \corral was configured with a time budget of 3000 seconds per tool invocation (one for each pair of entry points); \whoop did not use a timeout.

Our experiments use the \corral context switch bounds \texttt{/k:2}, \texttt{/k:5} and \texttt{/k:9}. A higher context switch bound would result into deeper interleavings being explored and thus a larger sequentialized program. Intuitively, this means that we would see even greater speedups using information from \whoop, when exploring deeper interleavings.

%When running \corral on its own, it can potentially report a spurious race (that \whoop does not report), because \corral is not enhanced with domain-specific information that \whoop uses to discard spurious races. Using \whoop + \corral, though, will inherently discard spurious data races (assuming a precise environmental model).

In Table~\ref{tab:results}, we can notice that \whoop is much faster than \corral. For example, \whoop analyses the dtlk char driver in 6.51 seconds, whereas \corral requires 1287.05 seconds.

% We did not discover any data races with larger context switch bounds.

%When \corral on its own tries to analyze the machzwd watchdog driver it explodes, because in some entry point pairs it attempts to analyze a large number of memory regions. For example for the entry point pair write vs init, it instruments interleavings at 26 different memory regions in a single invocation of \corral, which required approximately 1000 seconds to analyze. The same entry point pair is found race-free by \whoop. Using this information in \corral with the \yieldmr instrumentation, the pair takes less than a second to analyze. The \yieldcoarse instrumentation does not perform as well, because it works in the binary granularity of racy or non-racy entry points.