\newcommand{\colspacing}{\hspace{1.8em}}
\begin{table*}[t]
\small
\centering
\begin{tabular}{l rrr rrr rr r r}
\centering
& & & & \multicolumn{2}{c}{\textbf{\whoop}}
& \multicolumn{2}{c}{\textbf{\corral}}
& \multicolumn{2}{c}{\textbf{\whoop + \corral}}\\
\cmidrule(lr){5-6}
\cmidrule(lr){7-8}
\cmidrule(lr){9-10}

& & & & \multirow{2}{*}{\textbf{\#Racy}}
& & \multirow{2}{*}{\textbf{\#Races}}
& \multicolumn{1}{c}{\textbf{\yieldall}}
& \multicolumn{1}{c}{\textbf{\yieldcoarse}}
& \multicolumn{1}{c}{\textbf{\yieldmr}}\\
\cmidrule(lr){8-8}
\cmidrule(lr){9-9}
\cmidrule(lr){10-10}

\textbf{Benchmarks}
& \textbf{LoC}
& \textbf{\#MR}
& \textbf{\#Pairs}
& \textbf{Pairs}
& \textbf{\textbf{Time (s)}}
& \multicolumn{1}{r}{\textbf{Found?}}
& \multicolumn{1}{r}{\textbf{Time (s)}}
& \multicolumn{1}{r}{\textbf{Time (s)}}
& \multicolumn{1}{r}{\textbf{Time (s)}}\\[0.3em]

\toprule

generic\_nvram
& 152
& 25
& 14
& \multicolumn{1}{r}{6}
& \multicolumn{1}{r}{3.15}
& \multicolumn{1}{r}{\xmark}
& \multicolumn{1}{r}{191.36}
& \multicolumn{1}{r}{85.61}
& \multicolumn{1}{r}{34.39}\\

efirtc
& 420
& 35
& 6
& \multicolumn{1}{r}{2}
& \multicolumn{1}{r}{3.25}
& \multicolumn{1}{r}{\xmark}
& \multicolumn{1}{r}{202.31}
& \multicolumn{1}{r}{91.69}
& \multicolumn{1}{r}{20.36}\\

machzwd
& 457
& 49
& 10
& \multicolumn{1}{r}{9}
& \multicolumn{1}{r}{6.07}
& \multicolumn{1}{r}{1}
& \multicolumn{1}{r}{121.184}
& \multicolumn{1}{r}{125.55}
& \multicolumn{1}{r}{44.59}\\

intel\_scu\_wtchdg
& 568
& 25
& 10
& \multicolumn{1}{r}{5}
& \multicolumn{1}{r}{3.19}
& \multicolumn{1}{r}{1}
& \multicolumn{1}{r}{139.22}
& \multicolumn{1}{r}{78.73}
& \multicolumn{1}{r}{36.73}\\

ds1286
& 581
& 40
& 15
& \multicolumn{1}{r}{3}
& \multicolumn{1}{r}{4.51}
& \multicolumn{1}{r}{4}
& \multicolumn{1}{r}{553.48}
& \multicolumn{1}{r}{114.68}
& \multicolumn{1}{r}{31.60}\\

dtlk
& 656
& 37
& 21
& \multicolumn{1}{r}{10}
& \multicolumn{1}{r}{6.51}
& \multicolumn{1}{r}{\xmark}
& \multicolumn{1}{r}{239.517}
& \multicolumn{1}{r}{115.47}
& \multicolumn{1}{r}{79.05}\\

minix
& 690
& 34
& 90
& \multicolumn{1}{r}{50}
& \multicolumn{1}{r}{29.92}
& \multicolumn{1}{r}{20}
& \multicolumn{1}{r}{}
& \multicolumn{1}{r}{}
& \multicolumn{1}{r}{357.66}\\

aztcd
& 2496
& 85
& 14
& \multicolumn{1}{r}{13}
& \multicolumn{1}{r}{}
& \multicolumn{1}{r}{}
& \multicolumn{1}{r}{}
& \multicolumn{1}{r}{}
& \multicolumn{1}{r}{}\\

8139too
& 2694
& 
& 
& \multicolumn{1}{r}{}
& \multicolumn{1}{r}{}
& \multicolumn{1}{r}{}
& \multicolumn{1}{r}{}
& \multicolumn{1}{r}{}
& \multicolumn{1}{r}{}\\

r8169
& 7205
& 
& 
& \multicolumn{1}{r}{}
& \multicolumn{1}{r}{}
& \multicolumn{1}{r}{}
& \multicolumn{1}{r}{}
& \multicolumn{1}{r}{}
& \multicolumn{1}{r}{}\\[0.2em]
\end{tabular}
\caption{Data race analysis and scalability results.}
\label{tab:races}
\end{table*}

\subsection{Experimental Setup}
\label{eval:setup}

We conducted all experiments on ... We use the following software versions: Clang/LLVM 3.5; SMACK 1.4; Z3 4.1; \corral commit 3aa62d7425b5; Boogie commit d6a7f2bd79c9.

All reported times are in seconds and averages of N executions. We report: lines of C code (LoC); number of entry point pairs (\#Pairs); memory regions in the program (\#MRs); the number of racy memory regions that \whoop found (\#Racy MRs); the number of race reports that \whoop generated (\#Race Reports); the number of races that \corral found (same in all configurations).

\subsection{Benchmarks}
\label{eval:benchmarks}

We evaluate our methodology against 10 modules taken from the Linux kernel (currently a mix of latest and older distributions). \PDComment{Say something about the benchmarks}

\subsection{Context-Switch Granularity of Yield Instrumentation}
\label{eval:granularity}

We evaluate the scalability of our acceleration technique with respect to our benchmark sets.

Our experiments use the default \corral context switch bound \texttt{/k:2}. A higher context switch bound would result into deeper interleavings being explored and thus a larger sequentialised program. Intuitively, this means that we would see even greater speedups using information from \whoop, when exploring deeper interleavings. \PDComment{We should take a few measurements with higher k -- no need to be in the table -- just to verify this argument.}

When running \corral on its own, it can potentially report a spurious race (that \whoop does not report), because \corral is not enhanced with domain-specific information that \whoop uses to discard spurious races. Using \whoop + \corral, though, will inherently discard spurious data races (assuming a precise environmental model).

In Table~\ref{tab:races}, we can notice that \whoop is much faster than \corral. For example, \whoop analyses the dtlk char driver in 6.51 seconds, whereas \corral requires 1287.05 seconds.

When \corral on its own tries to analyse the machzwd watchdog driver it explodes, because in some entry point pairs it attempts to analyse a large number of memory regions. For example for the entry point pair write vs init, it instruments interleavings at 26 different memory regions in a single invocation of \corral, which required approximately 1000 seconds to analyse. The same entry point pair is found race-free by \whoop. Using this information in \corral with the \yieldmr instrumentation, the pair takes less than a second to analyse. The \yieldcoarse instrumentation does not perform as well, because it works in the binary granularity of racy or non-racy entry points.