\newcommand{\colspacing}{\hspace{1.8em}}
\begin{table}[t]
\small
\centering
\setlength{\tabcolsep}{0.3em}
\caption{Program statistics and race-checking results from running \whoop and \corral on our benchmarks.}
\label{tab:stats}
\begin{tabular}{l rrr rr r}
\centering
& & & & \multicolumn{2}{c}{\textbf{\whoop}}
& \textbf{\corral}\\
\cmidrule(lr){5-6}
\cmidrule(lr){7-7}

& & & & \multicolumn{1}{r}{\textbf{\#Racy}}
& \multicolumn{1}{r}{\textbf{\#Racy}}
& \multicolumn{1}{r}{\textbf{\#Races}}\\

\textbf{Benchmarks}
& \textbf{LoC}
& \textbf{\#Pairs}
& \textbf{\#MRs}
& \multicolumn{1}{r}{\textbf{Pairs}}
& \multicolumn{1}{r}{\textbf{MRs}}
& \multicolumn{1}{r}{\textbf{Found}}\\[0.3em]

\toprule

generic\_nvram
& 283
& 14
& 39
& \multicolumn{1}{r}{7}
& \multicolumn{1}{r}{2}
& 4\\

pc8736x\_gpio
& 354
& 27
& 55
& \multicolumn{1}{r}{13}
& \multicolumn{1}{r}{6}
& 5\\

machzwd
& 457
& 10
& 49
& \multicolumn{1}{r}{6}
& \multicolumn{1}{r}{3}
& 1\\

ssu100
& 568
& 7
& 27
& \multicolumn{1}{r}{\xmark}
& \multicolumn{1}{r}{\xmark}
& \xmark\\

intel\_scu\_wd
& 632
& 10
& 45
& \multicolumn{1}{r}{5}
& \multicolumn{1}{r}{1}
& 2\\

ds1286
& 635
& 15
& 49
& \multicolumn{1}{r}{5}
& \multicolumn{1}{r}{3}
& \xmark\\

dtlk
& 750
& 21
& 53
& \multicolumn{1}{r}{10}
& \multicolumn{1}{r}{6}
& \xmark\\

fs3270
& 883
& 15
& 54
& \multicolumn{1}{r}{9}
& \multicolumn{1}{r}{1}
& \xmark\\

gdrom
& 890
& 94
& 41
& \multicolumn{1}{r}{21}
& \multicolumn{1}{r}{2}
& \xmark\\

swim
& 996
& 28
& 80
& \multicolumn{1}{r}{15}
& \multicolumn{1}{r}{7}
& 8\\

intel\_nfcsim
& 1272
& 10
& 24
& \multicolumn{1}{r}{10}
& \multicolumn{1}{r}{2}
& \xmark\\

ps3vram
& 1499
& 4
& 32
& \multicolumn{1}{r}{1}
& \multicolumn{1}{r}{1}
& \xmark\\

sonypi
& 1729
& 30
& 62
& \multicolumn{1}{r}{19}
& \multicolumn{1}{r}{4}
& 2\\

sx8
& 1751
& 2
& 47
& \multicolumn{1}{r}{2}
& \multicolumn{1}{r}{1}
& 1\\

8139too
& 2694
& 46
& 37
& \multicolumn{1}{r}{40}
& \multicolumn{1}{r}{4}
& \xmark\\

r8169
& 7205
& 192
& 50
& \multicolumn{1}{r}{88}
& \multicolumn{1}{r}{1}
& \xmark\\

\bottomrule
\end{tabular}
\end{table}

\begin{table*}[t]
\small
\centering
\caption{Runtime comparison with different yield instrumentation granularities and context-switch bounds.}
\label{tab:results}
\begin{tabular}{l rrr rrr rr r r}
\centering
& \textbf{\whoop}
& \multicolumn{3}{c}{\textbf{\corral}}
& \multicolumn{6}{c}{\textbf{\whoop + \corral}}\\
\cmidrule(lr){2-2}
\cmidrule(lr){3-5}
\cmidrule(lr){6-11}

& \multirow{2}{*}{\textbf{Time}}
& \multicolumn{3}{c}{\textbf{\yieldall\xspace --- Time (sec)}}
& \multicolumn{3}{c}{\textbf{\yieldcoarse\xspace --- Time (sec)}}
& \multicolumn{3}{c}{\textbf{\yieldmr\xspace --- Time (sec)}}\\
\cmidrule(lr){3-5}
\cmidrule(lr){6-8}
\cmidrule(lr){9-11}

\textbf{Benchmarks}
& \textbf{\textbf{(sec)}}
& \multicolumn{1}{r}{\textbf{csb = 2}}
& \multicolumn{1}{r}{\textbf{csb = 5}}
& \multicolumn{1}{r}{\textbf{csb = 9}}
& \multicolumn{1}{r}{\textbf{csb = 2}}
& \multicolumn{1}{r}{\textbf{csb = 5}}
& \multicolumn{1}{r}{\textbf{csb = 9}}
& \multicolumn{1}{r}{\textbf{csb = 2}}
& \multicolumn{1}{r}{\textbf{csb = 5}}
& \multicolumn{1}{r}{\textbf{csb = 9}}\\[0.3em]

\toprule

generic\_nvram
& \multicolumn{1}{r}{2.7}
& \multicolumn{1}{r}{27.9}
& \multicolumn{1}{r}{38.4}
& \multicolumn{1}{r}{146.2}
& \multicolumn{1}{r}{17.6}
& \multicolumn{1}{r}{22.2}
& \multicolumn{1}{r}{90.8}
& \multicolumn{1}{r}{14.2}
& \multicolumn{1}{r}{16.5}
& \multicolumn{1}{r}{42.0}\\

pc8736x\_gpio
& \multicolumn{1}{r}{5.3}
& \multicolumn{1}{r}{145.6}
& \multicolumn{1}{r}{302.0}
& \multicolumn{1}{r}{}
& \multicolumn{1}{r}{89.5}
& \multicolumn{1}{r}{229.5}
& \multicolumn{1}{r}{}
& \multicolumn{1}{r}{41.7}
& \multicolumn{1}{r}{56.9}
& \multicolumn{1}{r}{426.6}\\

machzwd
& \multicolumn{1}{r}{}
& \multicolumn{1}{r}{}
& \multicolumn{1}{r}{}
& \multicolumn{1}{r}{}
& \multicolumn{1}{r}{}
& \multicolumn{1}{r}{}
& \multicolumn{1}{r}{}
& \multicolumn{1}{r}{}
& \multicolumn{1}{r}{}
& \multicolumn{1}{r}{}\\

ssu100
& \multicolumn{1}{r}{}
& \multicolumn{1}{r}{}
& \multicolumn{1}{r}{}
& \multicolumn{1}{r}{}
& \multicolumn{1}{r}{}
& \multicolumn{1}{r}{}
& \multicolumn{1}{r}{}
& \multicolumn{1}{r}{}
& \multicolumn{1}{r}{}
& \multicolumn{1}{r}{}\\

intel\_scu\_wd
& \multicolumn{1}{r}{}
& \multicolumn{1}{r}{}
& \multicolumn{1}{r}{}
& \multicolumn{1}{r}{}
& \multicolumn{1}{r}{}
& \multicolumn{1}{r}{}
& \multicolumn{1}{r}{}
& \multicolumn{1}{r}{}
& \multicolumn{1}{r}{}
& \multicolumn{1}{r}{}\\

ds1286
& \multicolumn{1}{r}{}
& \multicolumn{1}{r}{}
& \multicolumn{1}{r}{}
& \multicolumn{1}{r}{}
& \multicolumn{1}{r}{}
& \multicolumn{1}{r}{}
& \multicolumn{1}{r}{}
& \multicolumn{1}{r}{}
& \multicolumn{1}{r}{}
& \multicolumn{1}{r}{}\\

dtlk
& \multicolumn{1}{r}{}
& \multicolumn{1}{r}{}
& \multicolumn{1}{r}{}
& \multicolumn{1}{r}{}
& \multicolumn{1}{r}{}
& \multicolumn{1}{r}{}
& \multicolumn{1}{r}{}
& \multicolumn{1}{r}{}
& \multicolumn{1}{r}{}
& \multicolumn{1}{r}{}\\

fs3270
& \multicolumn{1}{r}{}
& \multicolumn{1}{r}{}
& \multicolumn{1}{r}{}
& \multicolumn{1}{r}{}
& \multicolumn{1}{r}{}
& \multicolumn{1}{r}{}
& \multicolumn{1}{r}{}
& \multicolumn{1}{r}{}
& \multicolumn{1}{r}{}
& \multicolumn{1}{r}{}\\

gdrom
& \multicolumn{1}{r}{}
& \multicolumn{1}{r}{}
& \multicolumn{1}{r}{}
& \multicolumn{1}{r}{}
& \multicolumn{1}{r}{}
& \multicolumn{1}{r}{}
& \multicolumn{1}{r}{}
& \multicolumn{1}{r}{}
& \multicolumn{1}{r}{}
& \multicolumn{1}{r}{}\\

swim
& \multicolumn{1}{r}{}
& \multicolumn{1}{r}{}
& \multicolumn{1}{r}{}
& \multicolumn{1}{r}{}
& \multicolumn{1}{r}{}
& \multicolumn{1}{r}{}
& \multicolumn{1}{r}{}
& \multicolumn{1}{r}{}
& \multicolumn{1}{r}{}
& \multicolumn{1}{r}{}\\

intel\_nfcsim
& \multicolumn{1}{r}{}
& \multicolumn{1}{r}{}
& \multicolumn{1}{r}{}
& \multicolumn{1}{r}{}
& \multicolumn{1}{r}{}
& \multicolumn{1}{r}{}
& \multicolumn{1}{r}{}
& \multicolumn{1}{r}{}
& \multicolumn{1}{r}{}
& \multicolumn{1}{r}{}\\

ps3vram
& \multicolumn{1}{r}{}
& \multicolumn{1}{r}{}
& \multicolumn{1}{r}{}
& \multicolumn{1}{r}{}
& \multicolumn{1}{r}{}
& \multicolumn{1}{r}{}
& \multicolumn{1}{r}{}
& \multicolumn{1}{r}{}
& \multicolumn{1}{r}{}
& \multicolumn{1}{r}{}\\

sonypi
& \multicolumn{1}{r}{}
& \multicolumn{1}{r}{}
& \multicolumn{1}{r}{}
& \multicolumn{1}{r}{}
& \multicolumn{1}{r}{}
& \multicolumn{1}{r}{}
& \multicolumn{1}{r}{}
& \multicolumn{1}{r}{}
& \multicolumn{1}{r}{}
& \multicolumn{1}{r}{}\\

sx8
& \multicolumn{1}{r}{}
& \multicolumn{1}{r}{}
& \multicolumn{1}{r}{}
& \multicolumn{1}{r}{}
& \multicolumn{1}{r}{}
& \multicolumn{1}{r}{}
& \multicolumn{1}{r}{}
& \multicolumn{1}{r}{}
& \multicolumn{1}{r}{}
& \multicolumn{1}{r}{}\\

8139too
& \multicolumn{1}{r}{}
& \multicolumn{1}{r}{}
& \multicolumn{1}{r}{}
& \multicolumn{1}{r}{}
& \multicolumn{1}{r}{}
& \multicolumn{1}{r}{}
& \multicolumn{1}{r}{}
& \multicolumn{1}{r}{}
& \multicolumn{1}{r}{}
& \multicolumn{1}{r}{}\\

r8169
& \multicolumn{1}{r}{228.1}
& \multicolumn{1}{r}{T.O.}
& \multicolumn{1}{r}{T.O.}
& \multicolumn{1}{r}{T.O.}
& \multicolumn{1}{r}{28755.2}
& \multicolumn{1}{r}{T.O.}
& \multicolumn{1}{r}{T.O.}
& \multicolumn{1}{r}{26939.4}
& \multicolumn{1}{r}{T.O.}
& \multicolumn{1}{r}{T.O.}\\

\bottomrule
\end{tabular}
\end{table*}

We performed experiments to validate the usefulness of the \whoop approach (see Section~\ref{whoop}). We first present race-checking results from running \whoop and \corral on \sizeOfBenchmarks drivers taken from the 4.0 Linux kernel distribution\footnote{https://www.kernel.org}. We then evaluate the runtime performance and scalability of \corral and \whoop + \corral with different yield instrumentation granularities and context-switch bounds, to show that \whoop can efficiently accelerate bug-finding with \corral.

\noindent
\textbf{Experimental Setup}\xspace\xspace All experiments were performed on a ... machine running ..., LLVM 3.5; SMACK 1.5; Z3 4.3.2; Boogie rev. 4192 and \corral rev. 534.

\noindent
\textbf{Benchmarks}\xspace\xspace We evaluate our methodology against \sizeOfBenchmarks drivers taken from the 4.0 Linux kernel. To thoroughly test \whoop against the Linux driver APIs, we chose drivers from many different domains: block; char; ethernet; near field communication (nfc); universal serial bus (usb); and watchdog. Further, we focused on choosing non-trivial drivers (e.g. large number of entry points, deep nesting of entry point helper functions, complex use of locking, use of function pointers, etc). We manually modeled the environment for all these drivers, a process that required more than a month of work.

\noindent
\textbf{Race-Checking}\xspace\xspace Table~\ref{tab:stats} presents statistics for all our benchmarks: lines of code (LoC); number of entry point pairs (\#Pairs); number of SMACK memory regions (\#MRs); number of racy pairs that \whoop identified (\#Racy Pairs); number of racy memory regions that \whoop reported (\#Racy MRs); and number of races that \corral discovered (\#Races Found) using a context-switch bound of 2.

We can see that, in general, \whoop reports (many) more races than \corral. This is expected because \whoop over-approximates the original driver for soundness: if a race exists, the tool will report it (assuming tool and model correctness), but \whoop can, and does, report false races. On the contrary, \corral reports only real races, but does that under pre-defined bounds and thus can miss races.

Most of the data races that \whoop and \corral found can be classified in two cases. The first case had to do with incrementing a global counter without using a lock. The reason behind this might be performance, but even benign races can lead to undefined behavior according to the C standard. The second case had to do with an entry point modifying a field of an object (either global or passed as parameter) without using a lock. This can easily lead to a data race if another concurrently running entry point accesses the same field of the same object. As an example of the second case, we discovered the following data race in the generic\_nvram char driver: during the \texttt{llseek} entry point, the driver is accessing the file offset \texttt{file->f\_pos} (\texttt{file} is passed as a parameter to the entry point) without first acquiring a lock. This can lead to a race if the driver invokes \texttt{llseek} from another thread passing the same \texttt{file} object as a parameter. When we investigated another char driver that uses the same API, we saw that it was protecting the offset access in the same entry point with a mutex. This made us suspicious that the race we found in generic\_nvram is real. We filed a bug report, but did not manage to get a response yet.

It is worthwhile to mention that we did not discover any additional data races with larger context switch bounds. This might mean that most races in these drivers require only a small amount of context-switches to manifest, or that \corral run out of resources before discovering the deeper bug.

\noindent
\textbf{Granularity of Context-Switches}\xspace\xspace Table~\ref{tab:results} presents runtime results from using \whoop, \corral and \whoop + \corral to analyze our benchmarks. All reported times are in seconds and averages of three runs. \corral was given a time budget of 8 hours (T.O. denotes a tool timeout) and a context-switch bound (csb) of 2, 5 and 9. By default, \corral instruments context-switches (i.e. \texttt{yield}) in all visible operations (denoted by \yieldall in the table). \whoop + \corral, instead, uses two different context-switch instrumentation granularities (see Section~\ref{whoop:bugfinding}): \yieldcoarse and \yieldmr.

A higher context switch bound results into deeper interleavings being explored and thus a larger sequentialized program. Intuitively, this means that we see even greater speedups using information from \whoop, when exploring deeper interleavings.

We can notice that \whoop is significantly faster than \corral. This is expected as \whoop achieves scalability using over-approximation and summarization techniques. This allows \whoop to perform especially well in complex drivers such as the r8169 ethernet driver: although \corral ... and timeouts with a csb of 9, \whoop manages to analyze the driver in 228.1 seconds. We believe that the reason behind this is that r8169 has many loops and uses deeply-nested recursion in some entry points, which \corral might not be able to handle efficiently.

%When running \corral on its own, it can potentially report a spurious race (that \whoop does not report), because \corral is not enhanced with domain-specific information that \whoop uses to discard spurious races. Using \whoop + \corral, though, will inherently discard spurious data races (assuming a precise environmental model).

%When \corral on its own tries to analyze the machzwd watchdog driver it explodes, because in some entry point pairs it attempts to analyze a large number of memory regions. For example for the entry point pair write vs init, it instruments interleavings at 26 different memory regions in a single invocation of \corral, which required approximately 1000 seconds to analyze. The same entry point pair is found race-free by \whoop. Using this information in \corral with the \yieldmr instrumentation, the pair takes less than a second to analyze. The \yieldcoarse instrumentation does not perform as well, because it works in the binary granularity of racy or non-racy entry points.