\subsection{Experimental Setup}
\label{eval:setup}

We conducted all experiments on ... We use the following software versions: Z3 4.1; SMACK 1.4; Clang/LLVM 3.5; \corral commit 3aa62d7425b5; Boogie commit d6a7f2bd79c9.

\subsection{Benchmarks}
\label{eval:benchmarks}

We evaluate our methodology against 10 modules taken from the Linux kernel (currently a mix of latest and older distributions).

\subsection{Granularity of Yield Instrumentation}
\label{eval:granularity}

We evaluate the scalability of our acceleration technique with respect to our benchmark sets.

Granularities: Yield-All (\corral without \whoop); \yieldcoarse; \yieldmr

\newcommand{\colspacing}{\hspace{1.8em}}
\begin{table*}[t]
\small
\centering
\begin{tabular}{l rrr rr rr r r}
\centering
& & & & \multicolumn{2}{c}{\textbf{\whoop}}
& \multicolumn{2}{c}{\textbf{\corral}}
& \multicolumn{2}{c}{\textbf{\whoop + \corral}}\\
\cmidrule(lr){5-6}
\cmidrule(lr){7-8}
\cmidrule(lr){9-10}

& & & & \multirow{2}{*}{\textbf{\#Racy}}
& & \multirow{2}{*}{\textbf{\#Races}}
& \multicolumn{1}{c}{\textbf{\yieldall}}
& \multicolumn{1}{c}{\textbf{\yieldcoarse}}
& \multicolumn{1}{c}{\textbf{\yieldmr}}\\
\cmidrule(lr){8-8}
\cmidrule(lr){9-9}
\cmidrule(lr){10-10}

\textbf{Benchmarks}
& \textbf{LoC}
& \textbf{\#MR}
& \textbf{\#Pairs}
& \textbf{Pairs}
& \textbf{\textbf{Time (s)}}
& \multicolumn{1}{r}{\textbf{Found?}}
& \multicolumn{1}{r}{\textbf{Time (s)}}
& \multicolumn{1}{r}{\textbf{Time (s)}}
& \multicolumn{1}{r}{\textbf{Time (s)}}\\[0.3em]

\toprule

generic\_nvram
& 152
& 25
& 14
& \multicolumn{1}{r}{6}
& \multicolumn{1}{r}{3.15}
& \multicolumn{1}{r}{\xmark}
& \multicolumn{1}{r}{191.36}
& \multicolumn{1}{r}{85.61}
& \multicolumn{1}{r}{34.39}\\

efirtc
& 420
& 35
& 6
& \multicolumn{1}{r}{2}
& \multicolumn{1}{r}{3.25}
& \multicolumn{1}{r}{\xmark}
& \multicolumn{1}{r}{202.31}
& \multicolumn{1}{r}{91.69}
& \multicolumn{1}{r}{20.36}\\

machzwd
& 457
& 49
& 10
& \multicolumn{1}{r}{9}
& \multicolumn{1}{r}{6.07}
& \multicolumn{1}{r}{1}
& \multicolumn{1}{r}{121.184}
& \multicolumn{1}{r}{125.55}
& \multicolumn{1}{r}{44.59}\\

intel\_scu\_wtchdg
& 568
& 25
& 10
& \multicolumn{1}{r}{5}
& \multicolumn{1}{r}{3.19}
& \multicolumn{1}{r}{1}
& \multicolumn{1}{r}{139.22}
& \multicolumn{1}{r}{78.73}
& \multicolumn{1}{r}{36.73}\\

ds1286
& 581
& 40
& 15
& \multicolumn{1}{r}{3}
& \multicolumn{1}{r}{4.51}
& \multicolumn{1}{r}{4}
& \multicolumn{1}{r}{553.48}
& \multicolumn{1}{r}{114.68}
& \multicolumn{1}{r}{31.60}\\

dtlk
& 656
& 37
& 21
& \multicolumn{1}{r}{10}
& \multicolumn{1}{r}{6.51}
& \multicolumn{1}{r}{\xmark}
& \multicolumn{1}{r}{239.517}
& \multicolumn{1}{r}{115.47}
& \multicolumn{1}{r}{79.05}\\

minix
& 690
& 34
& 90
& \multicolumn{1}{r}{50}
& \multicolumn{1}{r}{29.92}
& \multicolumn{1}{r}{20}
& \multicolumn{1}{r}{}
& \multicolumn{1}{r}{}
& \multicolumn{1}{r}{357.66}\\

aztcd
& 2496
& 85
& 14
& \multicolumn{1}{r}{13}
& \multicolumn{1}{r}{}
& \multicolumn{1}{r}{}
& \multicolumn{1}{r}{}
& \multicolumn{1}{r}{}
& \multicolumn{1}{r}{}\\

8139too
& 2694
& 
& 
& \multicolumn{1}{r}{}
& \multicolumn{1}{r}{}
& \multicolumn{1}{r}{}
& \multicolumn{1}{r}{}
& \multicolumn{1}{r}{}
& \multicolumn{1}{r}{}\\

r8169
& 7205
& 
& 
& \multicolumn{1}{r}{}
& \multicolumn{1}{r}{}
& \multicolumn{1}{r}{}
& \multicolumn{1}{r}{}
& \multicolumn{1}{r}{}
& \multicolumn{1}{r}{}\\[0.2em]
\end{tabular}
\caption{Results ...}
\label{tab:granularity}
\end{table*}

The machzwd watchdog driver explodes when running on vanilla \corral, because in some entry point pairs it attempts to analyse a large number of memory regions. For example for the entry point pair write vs init, it instruments interleavings at 26 different memory regions in a single invocation of \corral, which required approximately 1000 seconds to analyse. The same entry point pair is found race-free by \whoop. Using this information in \corral with the \yieldmr instrumentation, the pair takes less than a second to analyse. The \yieldcoarse instrumentation does not perform as well, because it works in the binary granularity of racy or non-racy entry points.