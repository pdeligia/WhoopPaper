Device drivers are complex pieces of system-level software, operating at the thin boundary between hardware and software to provide an interface between the operating system and hardware devices that are attached to a computer. Drivers are notoriously hard to develop and debug~\cite{corbet2005linux}. This has a negative impact on hardware product releases, as time-to-market is commonly dominated by driver development, verification and validation schedules~\cite{yavatkar2012era}.
%
Even after a driver has shipped, it typically has many undetected errors: Chou et al.~\cite{chou2001empirical} gathered data from 7 years of Linux kernel releases and found that the relative error-rate in driver source code is up to 10 times higher than in any other part of the kernel; Swift et al.~\cite{Swift2003windowsxp} found that 85\% of the system crashes in Windows XP are due to faulty drivers. Regarding \emph{concurrency bugs}, a recent study~\cite{ryzhyk2009dingo} found that they account for 19\% of the total bugs in Linux drivers, showcasing their significance. The majority of these concurrency bugs were found to be \emph{data races} or \emph{deadlocks} in various configuration functions and hot-plugging handlers.

Concurrency bugs are exacerbated by the complex and hostile environment in which drivers operate~\cite{corbet2005linux}. The OS can invoke drivers from multiple threads, a hardware device can issue interrupt requests that cause running processes to block and switch execution context, and the user may remove a device (hot-plugging) while some operation is still running.  These scenarios can cause \emph{data races} if insufficient synchronization mechanisms are in place to control concurrent access to shared resources.
%
Data races are a source of undefined behavior in C~\cite[p.\ 38]{iso/iec11}, and lead to nondeterministically occurring bugs that can be hard to reproduce, isolate and fix, especially in the context of complex operating systems.
%
Several techniques have been successfully used to analyze device drivers~\cite{ball2006thorough, clarke2004predicate, engler2000checking, henzinger2002temporal, cook2006termination, kuznetsov2010testing, renzelmann2012symdrive, lal2012corral}, but most focus on generic sequential program properties and protocol bugs. Linux kernel analyzers, such as sparse~\cite{corbet2004sparse}, coccinelle~\cite{padioleau2008doc} and lockdep~\cite{corbet2006lock}, can find deadlocks in kernel source code, but are unable to detect races. Techniques that can detect races in drivers~\cite{dawson2003racerx, qadeer2004kiss, pratikakis2006locksmith, voung2007relay, lal2012corral} are usually either \emph{unsound} (i.e.\ can miss real bugs) or \emph{imprecise} (i.e.\ can report false bugs), and typically sacrifice precision for scalability. Thus, there is a clear need for new tools that are able to detect races efficiently and precisely.

We present \whoop, an automated approach for static data race analysis in device drivers. \whoop is empowered by \emph{symbolic pairwise lockset analysis}, which attempts to prove a driver race-free by: (i) deriving a sound \emph{sequential} program that \emph{over-approximates} the originally concurrent program; (ii) instrumenting it to record \emph{locksets}; and (iii) using the locksets to assert that all accesses to the same shared resource are consistently protected by a common lock. Reducing analysis to reasoning over a sequential program avoids the need to enumerate thread interleavings, and allows reuse of existing successful sequential verification techniques.
%
We show that we can exploit the race-freedom guarantees provided by our symbolic analysis to achieve a sound partial-order reduction that significantly accelerates \corral~\cite{lal2012corral}, a precise bug-finder used by Microsoft to analyze drivers and other concurrent programs. Using \whoop and \corral we analyzed \sizeOfBenchmarks drivers from the Linux 4.0 kernel.  By combining \whoop and \corral we achieve analysis speedups in the range of 1.5--10$\times$ for most of our benchmarks, compared with using \corral in isolation.  For two drivers, we observed even greater speedups of 12$\times$ and 20$\times$.
%
\whoop currently supports Linux drivers, but our approach is conceptually generic and could be applied to analyze drivers for other operating systems as well as concurrent systems that use a similar programming model (e.g.\ file systems).

To summarize, our contributions are as follows:
\vspace{-0.5mm}
\begin{itemize}
\item We propose symbolic pairwise lockset analysis, a sound and scalable technique for automatically verifying the absence of data races in device drivers.
\vspace{-0.2mm}
\item We present \whoop, a tool that leverages our approach for analyzing data races in device drivers.
\vspace{-0.2mm}
\item We show that we can achieve a sound partial-order reduction using our technique to accelerate \corral, an industrial-strength bug-finder.
\vspace{-0.2mm}
\item We analyze \sizeOfBenchmarks Linux drivers, showing that \whoop is efficient at race-checking and accelerating \corral.
\end{itemize}
