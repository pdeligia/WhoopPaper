\whoop is a sound but imprecise static race analyzer. For developers who deem false alarms as unacceptable, we consider a method for leveraging the full or partial race-freedom guarantees provided by \whoop to accelerate \corral~\cite{lal2012corral}, a precise bug-finder used by Microsoft to analyze Windows drivers~\cite{lal2014powering}.  Because \corral operates on Boogie programs, it was easy to integrate it into our toolchain (see Fig.~\ref{fig:whoop}--C). Our technique, though, is general and capable in principle of accelerating any concurrency bug-finder that operates by interleaving threads at shared memory operations.

\corral is a symbolic bounded verifier for Boogie IVL that uses the Z3 SMT solver to statically reason about program behaviors. It checks for violations of provided assertions, and reports a precise counterexample if an assertion violation is found. \corral performs bounded exploration of a concurrent program in two steps. First, given a bound on the number of allowed context-switches, the concurrent program is appropriately \emph{sequentialized}, and the generated sequential version preserves reachable states of the original concurrent program up to the context bound~\cite{popl2011-eqr,cav2009-lqr,cavLalR08}. Then, \corral attempts to prove bounded (in terms of the number of loop iterations and recursion depth) sequential reachability of a bug in a goal-directed, lazy fashion to postpone state space explosion when analyzing a large program. It uses two key techniques to achieve this: (i) variable abstraction, where it attempts to identify a minimal set of shared variables that have to be precisely tracked in order to discharge all assertions; and (ii) stratified inlining, where it attempts to inline procedures on-demand as they are required for proving program assertions.

\noindent\textbf{Race-Checking Instrumentation }
%
To detect data races with \corral, \whoop outputs a Boogie program instrumented with a simple, but effective encoding of race checks. Whenever there is a write access to a shared variable $s$, \whoop instruments the program as follows:
%
\begin{boogie}
s = e;         // original write
yield;         // allow for a context-switch
assert s == e; // check written value
\end{boogie}%\vspace{-2mm}
%
Likewise, whenever there is a read access to $s$, \whoop instruments the program as follows:
%
\begin{boogie}
x = s;         // original read
yield;         // allow for a context-switch
assert x == s; // check read value
\end{boogie}%\vspace{-2mm}
%
A \texttt{yield} statement denotes a nondeterministic context-switch, and is used by \corral to guide the sequentialization.

\corral is inherently unsound (i.e.\ can miss real races), because it performs bounded verification. However, \corral is precise and, assuming a precise environmental model, it will only report true races. \whoop takes advantage of this precision to report only feasible data races.
%
Note that our instrumentation conveniently tolerates some benign races: it does not report a read-write race if the write access updates the racy memory location with the same value that already existed; it also does not report a write-write race if the two write accesses update the racy memory location with the same value, which can be different from the pre-existing.

In this work, we use \corral to analyze individual pairs of entry points. We do not use any abstraction to model additional threads, as we want \corral to report only true races. Because we only analyze pairs, though, \corral will miss races that require more than two threads to manifest. We could extend our setup so that more than two threads are considered by \corral, but because the number of threads that an OS kernel might launch is unknown in general, we are inevitably limited by some fixed maximum thread count.

\noindent\textbf{Sound Partial-Order Reduction }
%
By default, and assuming no race-freedom guarantees, \whoop instruments a \texttt{yield} after each shared memory access of each entry point, and after every lock and unlock operation.\footnote{We acknowledge that in the presence of data races and relaxed memory, even considering all interleavings of shared memory accesses may be insufficient to find all bugs.} \whoop then sends this instrumented program to \corral, which leverages sequentialization to explore all possible thread interleavings up to a pre-defined bound.
%
Our approach to accelerating \corral is simple and yet effective: if, thanks to \whoop's analysis, we know that a given statement that accesses shared memory cannot be involved in a data race, then we do not instrument a \texttt{yield} after this statement, and we also omit the \texttt{assert} that would check for a race.
%
This is a form of \emph{partial order reduction}~\cite{DBLP:books/sp/Godefroid96}, and reduces the number of context-switches that \corral must consider in a \emph{sound} manner: there is no impact on the bugs that will be detected.  This is because each access for which a \texttt{yield} is suppressed is guaranteed to be protected by some lock (a consequence of lockset analysis).  If the access is a write, its effects are not visible by the other entry point in the pair until the lock is released.  If the access is a read, the value of the shared location cannot change until the lock is released.  The fact that a \texttt{yield} is placed after each unlock operation suffices to take account of communication between entry points via the shared memory location.

We have implemented two different \texttt{yield} instrumentations in \whoop: \yieldcoarse and \yieldmr.
%
The first instruments \texttt{yield} statements in a binary fashion: if \whoop proves an entry point pair (EPP) as race-free, then it will instrument a \texttt{yield} only after each lock and unlock statement of the pair; else if \whoop finds that a pair might race, then it will instrument a \texttt{yield} after all visible operations of the pair.
%
\yieldmr is a finer-grained instrumentation: it instruments a \texttt{yield} only after each access to a memory region (MR) that might race in the pair (regardless if the pair has not been fully proven as race-free), and after each lock and unlock. In our experiments (see \S\ref{evaluation}), \yieldmr is significantly faster than \yieldcoarse.

Our partial-order reduction is able in principle to accelerate \corral for \emph{arbitrary} bug-finding in concurrent programs. Although we did preliminary explorations in this direction, identifying useful safety properties to check was challenging since drivers typically contain no assertions. Thus, in this paper we use \corral solely to find data races.
