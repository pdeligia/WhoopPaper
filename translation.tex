The main aim of this phase is to take an input Linux kernel module written in C, together with an environmental model, and output an over-approximated version of the program in Boogie IVL, a simple imperative language with well-defined verification-focused semantics used as the input to a number of cutting-edge verifiers (e.g. Boogie and \corral). To achieve this translation, \whoop uses three LLVM\footnote{http://llvm.org}-based tools: Chauffeur, Clang\footnote{http://clang.llvm.org} and SMACK\footnote{https://github.com/smackers/smack}~\cite{rakamaric2014smack}.

First, \whoop runs Chauffeur, a Clang-frontend that we developed to perform two tasks: (i) visit the abstract syntax tree (AST) of the original program and extract all available entry point identifier names in the given module, together with the identifier names of their corresponding kernel callbacks; and (ii) instrument the original program with a function that calls all available entry points. This function acts as the main entry point to the module after the latter has been registered with the kernel. This simple rewriting does not alter the semantics of the original program, is required so that Clang does not compile away any uncalled entry points (because we closed the environment using stub functions) and is also later used by \whoop to help instrument asynchronous calls to entry points and create checker procedures (see Sections~\ref{staticanalysis} and~\ref{bugfinding}). The output is a file that contains information regarding the entry points and a file with the rewritten program.

Next, the rewritten program is passed to Clang which compiles it to LLVM-IR~\cite{lattner2004llvm}, a low-level assembly-like language in single static assignment (SSA) form. SMACK then translates the LLVM-IR to Boogie IVL. SMACK leverages the pointer-alias analyses of LLVM to efficiently model the heap manipulation operations of the original program in Boogie IVL. To achieve scalability, SMACK is using a split-memory model instead of a monolithic one that would be difficult for backend verifiers to reason about~\cite{rakamaric2009scalable}.

\begin{lstlisting}[caption = Simple networking entry point in C, label = fig:original_program]
static void ep1(struct net_device *dev) {
  struct shared *tp = netdev_priv(dev);
  mutex_lock(&tp->wk.mutex);
  tp->resource = 4;
  mutex_unlock(&tp->wk.mutex);
}
\end{lstlisting}

Example~\ref{fig:original_program} shows a simple program in C and Example~\ref{fig:smack_translation} shows its corresponding translation in Boogie IVL using SMACK. The function \texttt{\$pa} models pointer arithmetic, while the map \texttt{\$M.0} is a memory region that typically denotes a shared resource. Notice that function calls (e.g. to lock and unlock a mutex) are preserved in the translation.

\begin{boogie}[caption = Translation to Boogie IVL using SMACK, label = fig:smack_translation]
procedure ep1(dev: int) modifies $M.0; {
  var $p0, $p1, $p2, $p3, $p4, $p5, $p6: int;
  $bb0:
    $p0 := dev;
    $p1 := $pa($p0, 32, 1);
    $p2 := $pa($pa($p1, 0, 24), 12, 1);
    $p3 := $pa($pa($p2, 0, 12), 0, 1);
    call mutex_lock($p3);
    $p4 := $pa($pa($p1, 0, 24), 4, 1);
    $M.0[$p4] := 4;
    $p5 := $pa($pa($p1, 0, 24), 12, 1);
    $p6 := $pa($pa($p5, 0, 12), 0, 1);
    call mutex_unlock($p6);
    return;
}
\end{boogie}